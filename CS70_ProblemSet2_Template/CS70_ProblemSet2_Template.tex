\documentclass{article}
\usepackage{amsmath,amssymb,amsthm}
\newtheorem*{prop}{Proposition}
\renewcommand{\theenumi}{\alph{enumi}}
\usepackage[shortlabels]{enumitem}
\usepackage[nobreak=true]{mdframed}
\usepackage{tikz}
\usetikzlibrary{matrix,calc}
\usepackage{color}
\usepackage{chngpage}
\usepackage[margin=0.75in]{geometry}

\title{Problem Set 2}
\author{Name: $\quad$SID: }
\date{Spring 2016$\quad$GSI: }
\begin{document}
\maketitle

\subsection*{1. Matrices}
Use induction to prove that for all positive integers $n$, all of the entries in the matrix
$$\left(\begin{array}{cc}1 & 0 \\3 & 1\end{array}\right)^n$$
are $\leqslant 3n$.
\begin{mdframed}
\textbf{Solution}
% Solution here

\end{mdframed}

\clearpage

\subsection*{2. Divergence of Harmonic Series}
You may have seen the series $1+\frac{1}{2}+\frac{1}{3}+\ldots$ in calculus. This is known as a \textit{harmonic series}, and it diverges, i.e. the sum approaches infinity. We are going to prove this fact using induction.

\noindent{Let} $$H_j=\sum_{k=1}^{j}\frac{1}{k}$$ Use mathematical induction to show that, for all integers $n\geqslant 0, H_{2^n}\geqslant 1+\frac{n}{2}$, thus showing that $H_j$ must grow unboundedly as $j\to\infty$.
\begin{mdframed}
\textbf{Solution}
% Solution here

\end{mdframed}

\clearpage

\subsection*{3. Finitely Many Solutions in $\frac{1}{n_1}+\ldots+\frac{1}{n_k}=r$}
Prove that for every positive integer $k$, the following is true:

\vspace{2mm}
For every real number $r>0$, there are only finitely many solutions in positive integers $\frac{1}{n_1}+\ldots+\frac{1}{n_k}=r$.
\vspace{2mm}

\noindent{In other words, there exists some number $m$ (that depends on $k$ and $r$) such that there are at most $m$ ways of choosing a positive integer $n_1$, and a (possibly different) positive integer $n_2$, etc., that satisfy the equation.}

\vspace{1mm}
\noindent{Hint: You may assume $n_1\leqslant n_2 \leqslant\ldots\leqslant n_k$ without losing generality. (Why? Think about it.)}
\begin{mdframed}
\textbf{Solution}
% Solution here

\end{mdframed}

\clearpage

\subsection*{4. Objective Preferences}
Imagine that in the context of stable marriage all men have the same preference list. That is to say there is a global ranking of women, and men's preferences are directly determined by that ranking.
\begin{enumerate}
\item Prove that the first woman in the ranking has to be paired with her first choice in any stable pairing.
\begin{mdframed}
\textbf{Solution}
% Solution here

\end{mdframed}
\item Prove that the second woman has to be paired with her first choice if that choice is not the same as the first woman's first choice. Otherwise she has to be paired with her second choice.
\begin{mdframed}
\textbf{Solution}
% Solution here

\end{mdframed}
\item Continuing this way, assume that we have determined the pairs for the first $k -1$ women in the
ranking. Who should the $k$-th woman be paired with?
\begin{mdframed}
\textbf{Solution}
% Solution here

\end{mdframed}
\item Prove that there is a unique stable pairing.
\begin{mdframed}
\textbf{Solution}
% Solution here

\end{mdframed}
\end{enumerate}


\clearpage

\subsection*{5. TA Problem}
You have been asked to assign TAs for the fall semester. Each class has its own method for ranking candidates, and each candidate has their own preferences. An assignment is \textbf{\underline{unstable}} if a class and a candidate prefer each other to their current assignments. Otherwise, it is \textbf{\underline{stable}}.

\vspace{3mm}

\noindent{Candidate information:}
\begin{center}
\begin{tabular}{ |c||c|c|c|c|c|c|c| } 
\hline
 Candidate & CS61C Grade & CS70 & CS61A Grade & Teaching Experience & Overall GPA & Preferences \\ 
  \hline
A & A+ & A    & A   & Yes & 3.80 & CS61C$\,>\,$CS70$\,>\,$CS61A \\
B & A   & A    & A   & No  & 3.61 & CS61C$\,>\,$CS61A$\,>\,$CS70 \\
C & A   & A+ & A-  & Yes & 3.60 & CS61C$\,>\,$CS70$\,>\,$CS61A \\
\hline
\end{tabular}
\end{center}

\vspace{3mm}
Ranking Method:
\begin{itemize}
\item CS61C: Rank by CS61C grade. Break ties using teaching experience, then overall GPA.
\item CS70: Rank by teaching experience. Break ties using CS70 grade, then overall GPA.
\item CS61A: Rank by CS61A grade. Break ties using overall GPA, then teaching experience.
\end{itemize}

\begin{adjustwidth}{-0.1in}{}
\noindent{\begin{enumerate}

\item Find a stable assignment.
\begin{mdframed}
\textbf{Solution}
% Solution here

\end{mdframed}

\item Can you find another, or is there only one stable assignment (if there is only one, why)?
\begin{mdframed}
\textbf{Solution}
% Solution here

\end{mdframed}
\begin{adjustwidth}{-0.195in}{-4in}{CS61C is overenrolled and needs two TAs. There is another candidate.}\end{adjustwidth}
\begin{center}
\begin{adjustwidth}{-0.195in}{-1in}
\begin{tabular}{ |c||c|c|c|c|c|c|c| } 
\hline
 Candidate & CS61C Grade & CS70 & CS61A Grade & Teaching Experience & Overall GPA & Preferences \\ 
  \hline
D&A+&A&A+&No&3.90&CS70$\,>\,$CS61A$\,>\,$CS61C \\
\hline
\end{tabular}
\end{adjustwidth}
\end{center}
\item Find a stable assignment
\begin{mdframed}
\textbf{Solution}
% Solution here

\end{mdframed}
\item Prove your assignment in Part (c) is stable.
\begin{mdframed}
\textbf{Solution}
% Solution here

\end{mdframed}

\end{enumerate}}
\end{adjustwidth}

\clearpage

\subsection*{6. Better Off Alone}
In the stable marriage problem, suppose that some men and women have standards and would not just settle for anyone. In other words, in addition to the preference orderings they have, they prefer being alone to being with some of the lower-ranked individuals (in their own preference list). A pairing could ultimately have to be partial, i.e., some individuals would remain single.

\vspace{1mm}
\noindent{The notion of stability here should be adjusted a little bit. A pairing is stable if}
\begin{itemize}
\item there is no paired individual who prefers being single over being with his/her current partner,
\item there is no paired man and paired woman that would both prefer to be with each other over their
current partners, and
\item there is no single man and single woman that would both prefer to be with each other over being
single.
\item there is no paired man and single woman (or single man and paired woman) that would both prefer to be with each other over the current choice (the current partner or being alone).
\end{itemize}
\begin{adjustwidth}{-0.1in}{}
\begin{enumerate}
\item Prove that a stable pairing still exists in the case where we allow single individuals. You can approach this by introducing imaginary mates that people "marry" if they are single. How should you adjust the preference lists of people, including those of the newly introduced imaginary ones for this to work?
\begin{mdframed}
\textbf{Solution}
% Solution here

\end{mdframed}

\item  As you saw in the lecture, we may have different stable pairings. But interestingly, if a person remains single in one stable pairing, s/he must remain single in any other stable pairing as well (there really is no hope for some people!). Prove this fact by contradiction.
\begin{mdframed}
\textbf{Solution}
% Solution here

\end{mdframed}

\end{enumerate}
\end{adjustwidth}

\clearpage


\end{document}