\documentclass{article}\usepackage{amsmath,amssymb,amsthm,tikz,tkz-graph,color,chngpage,soul,hyperref,csquotes,graphicx,floatrow}\newcommand*{\QEDB}{\hfill\ensuremath{\square}}\newtheorem*{prop}{Proposition}\renewcommand{\theenumi}{\alph{enumi}}\usepackage[shortlabels]{enumitem}\usepackage[nobreak=true]{mdframed}\usetikzlibrary{matrix,calc}\MakeOuterQuote{"}\usepackage[margin=0.75in]{geometry} \newtheorem{theorem}{Theorem}

\title{Problem Set 5}
\author{Name: $\quad$SID: }
\date{Spring 2016$\quad$GSI: }
\begin{document}
\maketitle

%%%% Problem 1 %%%%

\subsection*{1. Proof Practice}
The purpose of this problem is to practice formally proving a statement, when you intuitively "know" why it's true.

\vspace{3mm}

\noindent Suppose that there are $n$ chickens in a farmyard. Chickens are rather aggressive birds that tend to establish dominance in relationships by pecking. (Hence the term "pecking order".) In particular, for each pair of distinct chickens, either the first pecks the second or the second pecks the first, but not both. We say that chicken $u$ virtually pecks chicken $v$ if either:
\begin{itemize}
\item Chicken $u$ directly pecks chicken $v$, or
\item Chicken $u$ pecks some other chicken $w$ who in turn pecks chicken $v$.
\end{itemize}
A chicken that virtually pecks every other chicken is called a \textit{king chicken}.
We can model this situation with a tournament digraph. The vertices are chickens, and an edge $u\to v$
indicates that chicken $u$ pecks chicken $v$. Notice that there could be multiple kings.

\begin{theorem}The chicken with the largest outdegree in an $n$-chicken tournament is a king.\end{theorem}

\noindent Intuitively the theorem is true because if the chicken with the largest outdegree was not a king then there would be a chicken that pecks everyone that the fake king pecks, as well as the fake king, i.e. have an even larger outdegree.

\vspace{3mm}

\noindent Turn this intuition into a formal proof.
\begin{mdframed}
\textbf{Solution}
% Solution here

\end{mdframed}

\clearpage

%%%% Problem 2 %%%%

\subsection*{2. What's in a googolplex?}
A "googolplex", the namesake of Google's "Googleplex" headquarters, is the number written as $1$ followed by $10^{100}$ zeroes. That is, it's $10^{{10}^{100}}$. For a positive integer $n$, we define "$n$ factorial" (written $n!$) as the product of all positive integers from $1$ to $n$, i.e. $n\cdot (n-1)\cdot\ldots\cdot 2\cdot 1$. As you might imagine, $10^{{10}^{100}}!$, googleplex factorial, is a Very Large Number. Let:
\begin{align*} m \quad= \quad & 51859259867354235424444672378388838622134534634634534562 \\ & 43434534634420981645243591345918100075131114594357234591 \\ & 10235098237457523423457591117449042203525117456777989031 \\ \end{align*}
Note that that's a single $168$-digit number (it just doesn't fit onto one line). Calculate what $10^{{10}^{100}}!$ is congruent to, modulo $m$. That is, find the value of:
$$10^{{10}^{100}}!\bmod m$$
Show all your work. Hint: if your answer takes more than 5-10 lines of text, you are probably doing it wrong.
\begin{mdframed}
\textbf{Solution}
% Solution here

\end{mdframed}
\clearpage

%%%% Problem 3 %%%%

\subsection*{3. Polynomial Interpolations}
\begin{enumerate}
\item Consider the set of four points $\{(0,1),(1,-2),(3,4),(4,0)\}$, construct the unique degree-$3$ polynomial (over the reals) that passes through these four points by writing down and solving a system of linear equations.
\begin{mdframed}
\textbf{Solution}
% Solution here

\end{mdframed}
\item Use Lagrange interpolation to find a polynomial $p(x)$ of degree at most $2$ that passes through the points $(1,2),(2,3),$ and $(3,5)$, working in $GF(7)$. In other words, we want $p(x)$ to satisfy $p(1)\equiv 2\pmod 7$, $p(2)\equiv 3\pmod 7$, and $p(3)\equiv 5\pmod 7$. Show your work clearly and use the same notations as in Lecture Note 8.
\begin{mdframed}
\textbf{Solution}
% Solution here

\end{mdframed}
\end{enumerate}
\clearpage

%%%% Problem 4 %%%%

\subsection*{4. Secret Sharing}
Suppose we wish to share a secret among five people, and we decide to work modulo $7$. We construct a degree-two polynomial $q(x)=ax^2+bx+s$ by picking the coefficients $a$ and $b$ at random (mod $7$); the constant term is the secret $s$ (also a number mod $7$). Now suppose that three of the people get together and share the information that $q(1)=5, q(2)=2,$ and $q(4)=2$. Use Lagrange interpolation to find the polynomial $q$ and the secret $s$. Show all your work.
\begin{mdframed}
\textbf{Solution}
% Solution here

\end{mdframed}
\clearpage

%%%% Problem 5 %%%%

\subsection*{5. Properties of $GF(p)$}
\begin{enumerate}
\item Show that if $p(x)$ and $q(x)$ are polynomials over the reals (or complex, or rationals) and $p(x)\cdot q(x)=0$ for all $x$, then either $p(x)=0$ for all $x$ or $q(x)=0$ for all $x$ or both.
\begin{mdframed}
\textbf{Solution}
% Solution here

\end{mdframed}
\item Show that the claim in part (a) is false for finite fields $GF(p)$.
\begin{mdframed}
\textbf{Solution}
% Solution here

\end{mdframed}
\end{enumerate}
\clearpage
\end{document}