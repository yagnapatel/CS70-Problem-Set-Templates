\documentclass{article}\usepackage{amsmath,amssymb,amsthm,tikz,tkz-graph,color,chngpage,soul,hyperref,csquotes,graphicx,floatrow}\newcommand*{\QEDB}{\hfill\ensuremath{\square}}\newtheorem*{prop}{Proposition}\renewcommand{\theenumi}{\alph{enumi}}\usepackage[shortlabels]{enumitem}\usepackage[nobreak=true]{mdframed}\usetikzlibrary{matrix,calc}\MakeOuterQuote{"}\usepackage[margin=0.75in]{geometry}

\title{Problem Set 4}
\author{Name: $\quad$SID: }
\date{Spring 2016$\quad$GSI: }
\begin{document}
\maketitle

%%%% Problem 1 %%%%

\subsection*{1. Amaze Your Friends!}
\begin{enumerate}
\item You want to trick your friends into thinking you can perform mental arithmetic with very large numbers What are the last digits of the following numbers?
\begin{enumerate}[i.]
\item $11^{2014}$
\begin{mdframed}
\textbf{Solution}
% Solution here

\end{mdframed}
\item $9^{10001}$
\begin{mdframed}
\textbf{Solution}
% Solution here

\end{mdframed}
\item $3^{987654321}$
\begin{mdframed}
\textbf{Solution}
% Solution here

\end{mdframed}
\end{enumerate}
\item You know that you can quickly tell a number $n$ is divisible by $9$ if and only if the sum of the digits of $n$ is divisible by $9$. Prove that you can use this trick to quickly calculate if a number is divisible by $9$.
\begin{mdframed}
\textbf{Solution}
% Solution here

\end{mdframed}
\end{enumerate}
\clearpage

%%%% Problem 2 %%%%

\subsection*{2. Short Answer: pmodular Arithmetic}
\begin{enumerate}
\item What is the multiplicative inverse of $3\pmod 7$?
\begin{mdframed}
\textbf{Solution}
% Solution here

\end{mdframed}
\item What is the multiplicative inverse of $n-1$ pmodulo $n$? (An expression that may involve $n$. Simplicity matters.)
\begin{mdframed}
\textbf{Solution}
% Solution here

\end{mdframed}
\item What is the solution to the equation $3x=6\pmod 17$? (A number in $\{0,\ldots,16\}$ or "No solution".)
\begin{mdframed}
\textbf{Solution}
% Solution here

\end{mdframed}
\item Let $R_0=0;R_1=2;R_n=4R_{n-1}-3R_{n-2}$ for $n\geqslant 2$. Is $R_n=2\pmod 3$ for $n\geqslant 1$? (True or False)
\begin{mdframed}
\textbf{Solution}
% Solution here

\end{mdframed}
\item Given that \textit{extended} $-\gcd(53,m)=(1,7,-1)$, that is $(7)(53)+(-1)m=1$, what is the solution \\ to $53x+3=10\pmod m$? (Answer should be an expression that is interpreted $\pmod m$, and shouldn't consist of fractions.)
\begin{mdframed}
\textbf{Solution}
% Solution here

\end{mdframed}
\end{enumerate}
\clearpage

%%%% Problem 3 %%%%

\subsection*{3. Combining Moduli}
Suppose we wish to work modulo $n=40$. Note that $40=5\times 8$, with $\gcd(5,8)=1$. We will show that in many ways working modulo $40$ is the same as working modulo $5$ and modulo $8$, in the sense that instead of writing down $c\pmod {40}$, we can just write down $c\pmod 5$ and $c\pmod 8$.
\begin{enumerate}
\item What is $8\pmod 5$ and $8\pmod 8$? Find a number $a\pmod {40}$ such that $a\equiv 1\pmod 5$ and $a\equiv 0\pmod 8$.
\begin{mdframed}
\textbf{Solution}
% Solution here

\end{mdframed}
\item Now find a number $b\pmod {40}$ such that $b\equiv 0\pmod 5$ and $b\equiv 1\pmod 8$.
\begin{mdframed}
\textbf{Solution}
% Solution here

\end{mdframed}
\item Now suppose you wish to find a number $c\pmod {40}$ such that $c\equiv 2\pmod 5$ and $c\equiv 5\pmod 8$. Find $c$ by expressing it in terms of $a$ and $b$.
\begin{mdframed}
\textbf{Solution}
% Solution here

\end{mdframed}
\item Repeat to find a number $d\pmod {40}$ such that $d\equiv 3\pmod 5$ and $d\equiv 4\pmod 8$.
\begin{mdframed}
\textbf{Solution}
% Solution here

\end{mdframed}
\item Compute $c\times d\pmod {40}$. Is it true that $c\times d\equiv 2\times 3\pmod 5$, and $c\times d\equiv 5\times 4\pmod 8$?
\begin{mdframed}
\textbf{Solution}
% Solution here

\end{mdframed}
\end{enumerate}
\clearpage

%%%% Problem 4 %%%%

\subsection*{4. The Last Digit}
Let $a$ be a positive integer. Consider the following sequence of numbers $x$ defined by:
\begin{align*}
x_0 &= a \\
x_n &= x_{n-1}^2+x_{n-1}+1\text{  if  }n>0 \\
\end{align*}
\begin{enumerate}
\item Show that if the last digit of $a$ is $3$ or $7$, then for every $n$, the last digit of $x_n$ is respectively $3$ or $7$.
\begin{mdframed}
\textbf{Solution}
% Solution here

\end{mdframed}
\item Show that there exists $k>0$ such that the last digit of $x_n$ for $n\geqslant k$ is constant. Give the smallest possible $k$, \textit{no matter what} $a$ is.
\begin{mdframed}
\textbf{Solution}
% Solution here

\end{mdframed}
\end{enumerate}
\clearpage

%%%% Problem 5 %%%%

\subsection*{5. Euclid's Extended GCD Algorithm}
\begin{enumerate}
\item Compute the inerse of $37$ modulo $64$ using Euclid's extended GCD algorithm.
\begin{mdframed}
\textbf{Solution}
% Solution here

\end{mdframed}
\item Prove that $\gcd\left(F_n,F_{n-1}\right)=1$, where $F_0=0$ and $F_1=1$ and $F_n=F_{n-1}+F_{n-2}$.
\begin{mdframed}
\textbf{Solution}
% Solution here

\end{mdframed}
\end{enumerate}
\clearpage

%%%% Problem 6 %%%%

\subsection*{6. Bijections}
Let $n$ be an odd number. Let $f(x)$ be a function from $\{0,1,\ldots,n-1\}$ to $\{0,1,\ldots,n-1\}$. In each of these cases say whether or not $f(x)$ is a bijection. Justify your answer (either prove $f(x)$ is a bijection or give a counterexample).
\begin{enumerate}
\item $f(x)=2x\pmod n$
\begin{mdframed}
\textbf{Solution}
% Solution here

\end{mdframed}
\item $f(x)=5x\pmod n$
\begin{mdframed}
\textbf{Solution}
% Solution here

\end{mdframed}
\item $n$ is prime and $$f(x)=\begin{cases} 0 & \text{if }x= 0 \\ x^{-1}\pmod n & \text{if }x\ne 0 \end{cases}$$
\begin{mdframed}
\textbf{Solution}
% Solution here

\end{mdframed}
\item $n$ is prime and $f(x)=x^2\pmod n$.
\begin{mdframed}
\textbf{Solution}
% Solution here

\end{mdframed}
\end{enumerate}
\clearpage

%%%% Problem 7 %%%%

\subsection*{7. Using RSA}
Kevin and Bob decide to apply the RSA cryptography so that Kevin can send a secret message to Bob.
\begin{enumerate}
\item Assuming $p=3,q=11,$ and $e=7$, what is $d$? Calculate the exact value.
\begin{mdframed}
\textbf{Solution}
% Solution here

\end{mdframed}
\item Following Part (a), what is the original message if Bob receives $4$? Calculate the exact value.
\begin{mdframed}
\textbf{Solution}
% Solution here

\end{mdframed}
\end{enumerate}
\clearpage

%%%% Problem 8 %%%%

\subsection*{8. Tweaking RSA}
(This problem will not be graded, the solution will be posted on the problem thread on piazza.)
\begin{enumerate}
\item You are trying to send a message to your friend, and as usual, Eve is trying to decipher what the message is. However, you get lazy, so you use $N=p$, and $p$ is prime. Similar to the original method, for any message $x\in \{0,1,\ldots,N-1\}, E(x)\equiv x^e\pmod N$, and $D(y)\equiv y^d\pmod N$. Show how you choose $e$ and $d$ in the encryption and decryption function, respectively. Prove that the message $x$ is recovered after it goes through your new encryption and decryption functions, $E(x)$ and $D(y)$
\begin{mdframed}
\textbf{Solution}
% Solution here

\end{mdframed}
\item Can Eve now compute $d$ in the decryption function? If so, by what algorithm?
\begin{mdframed}
\textbf{Solution}
% Solution here

\end{mdframed}
\item Now you wonder if you can modify the RSA encryption method to work with three primes ($N=pqr$ where $p,q,r$ are all prime). Explain how you can do so.
\begin{mdframed}
\textbf{Solution}
% Solution here

\end{mdframed}
\end{enumerate}
\clearpage
\end{document}